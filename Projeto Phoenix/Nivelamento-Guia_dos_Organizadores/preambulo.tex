\documentclass[12pt,a4paper]{article}

% Configurando margens da página
\usepackage[headheight=1.5 cm, top=3.5 cm, bottom=2.0 cm, left=3.0 cm, rigth=2.0 cm]{geometry}
\usepackage{setspace}

% Definindo a lingauem e a fonte do texto
\usepackage[T1]{fontenc}
\usepackage[utf8]{inputenc}
\usepackage[brazil]{babel}

% Importando os pacotes de matemática
\usepackage{amsthm}
\usepackage{amsmath}
\usepackage{mathrsfs}
\usepackage{amsfonts} %Configurações para letras dos conjuntos numéricos
\usepackage{mathdots}
\usepackage{cancel}

% Pacote para habilitar múltiplas colunas
\usepackage{multicol}
\setlength{\columnsep}{0.8cm}

% Configurando o recuo da primeira linha
\usepackage{indentfirst}
\setlength{\parindent}{1.25cm}

% Para incluir figuras
\usepackage[dvips]{graphicx}
\usepackage[usenames,dvipsnames]{xcolor}
\usepackage{transparent}
\usepackage{subcaption}

% Para ajustar tabelas
\usepackage{array}
\renewcommand{\arraystretch}{1.7}% Ajuste a altura das células

%Para navegação pelo documento
\usepackage{hyperref}
\hypersetup{
	colorlinks,
	linkcolor=main,
	anchorcolor=red,
	citecolor=blue!80,
	urlcolor=blue!80,
}

%Configurações de cabeçalho
\usepackage{fancyhdr}
\fancyhf{}
\renewcommand{\headrulewidth}{0.5pt}
\fancyhead[l]{\includegraphics[height=1.5cm]{source/logo-latex.png}}
\rhead{
\includegraphics[height=1.5cm]{source/niv_logo_FI.png}
\includegraphics[height=1.5cm]{source/niv_logo_PG.png}
\includegraphics[height=1.5cm]{source/niv_logo_QI.png}
\includegraphics[height=1.5cm]{source/niv_logo_BIO.png}
\includegraphics[height=1.5cm]{source/niv_logo_PC.png}
\includegraphics[height=1.5cm]{source/niv_logo_IB.png}
}
\rfoot{\thepage}

% Define um novo estilo de cabeçalho
\fancypagestyle{mystyle}{
  \fancyhf{}  % Limpa todos os cabeçalhos e rodapés anteriores
  \renewcommand{\headrulewidth}{0.0pt}
  \fancyhead[L]{\includegraphics[height = 1.2 cm]{source/itec-ufpa-logo.png}}
  \fancyhead[C]{\includegraphics[height = 1.5 cm]{source/logo-latex.png}}
  \fancyhead[R]{\includegraphics[height = 1.2 cm]{source/tutotia.jpg}}
  \fancyfoot[R]{\thepage}  % Número da página no centro do rodapé
}

\pagestyle{fancy}


%Criando o ambiente para exercícios
\theoremstyle{definition}
%\newtheorem{Que}{Quest\~ao}
\usepackage{lipsum} 
\usepackage{enumerate}
\usepackage{enumitem}


%Caixa de texto
\usepackage[most]{tcolorbox}
\usepackage{varwidth}

\definecolor{main}{HTML}{006b8f}
\definecolor{azul1}{HTML}{DAF2FF}
\definecolor{azul3}{HTML}{ADDDF0}
\definecolor{azul2}{rgb}{0.36,0.54, 0.66}
\definecolor{preto}{HTML}{3F3F3F}
\definecolor{aqua1}{HTML}{D8FEF1}
\definecolor{aqua2}{HTML}{AAE4CF}
\definecolor{lavanda1}{HTML}{D6D4F2}
\definecolor{lavanda2}{HTML}{B4B1E2}
\definecolor{cinza1}{HTML}{F0F0F0}
\definecolor{cinza2}{HTML}{E0E0E0}
\definecolor{verticalbarcolor}{RGB}{0,102,204}
\definecolor{verde}{HTML}{D3EBDB}


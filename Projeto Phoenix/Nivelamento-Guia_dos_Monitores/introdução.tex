\section{Introdução}
Olá monitor! \\

Seja bem-vindo à de Oficina de Nivelamento! Este guia foi elaborado para fornecer orientações fundamentais, principalmente aos nossos novos monitores, já que vocês desempenham um papel importante no apoio ao sucesso acadêmico dos nossos alunos.

A função de monitoria não apenas oferece uma oportunidade única para aprimorar suas habilidades acadêmicas, mas também desafia você a se tornar um mentor valioso para os seus colegas. Neste guia, você encontrará informações essenciais sobre suas responsabilidades, dicas práticas para um desempenho eficaz e uma visão geral dos recursos disponíveis.

Acreditamos que a sua dedicação e entusiasmo desempenharão um papel vital na criação de um ambiente acadêmico positivo e colaborativo. Ao seguir as diretrizes apresentadas neste guia, você não apenas contribuirá para o sucesso dos seus colegas, mas também terá uma experiência gratificante e enriquecedora como monitor.

Agradecemos por sua disposição em aceitar este papel e desejamos a você muito sucesso em sua jornada como monitor!

\section{Sobre o Nivelamento}
A oficina de Nivelamento surgiu em 2023 como uma iniciativa dos centros acadêmicos de engenharia com o objetivo de suprir a falta que o PCNA provocava.

\section{Organização e Dinâmica das aulas}
Cada grupo é formado por pelo menos um monitor e um ministrante, que são alocados para uma turma de 40 a 60 alunos. Formalmente é definido que o ministrante é o responsável por expor o conteúdo teórico e controlar o ritmo da aula ao passo que o monitor se mantém disponível para atendimento de forma mais individual e para eventuais resolução de exercícios em quadro. Entretanto, cada equipe de ministrante e monitor tem a liberdade.

\newpage
\section{Dicas para um monitor de Sucesso}
De forma direta, não existe receita mágica para ser um bom monitor, apenas o tempo e a experiencia são capazes de aprimorar seu desempenho como tal. Mas existem algumas regras, dicas e sugestões que certamente irão lhe ajudar a ter um bom começo no desempenho deste papel. Algumas dela são:



\begin{itemize}[label = \textcolor{main}{\textbullet}]
    \item Garanta que todos os alunos sintam-se bem-vindos e respeitados.

    \item Mantenha uma comunicação clara e aberta com os alunos.
    
    \item Esteja disposto a ouvir e a responder perguntas de forma compreensível.

    \item Esteja entusiasmado e positivo em relação ao seu papel.
    
    \item Encoraje os alunos a enfrentar desafios com uma mentalidade positiva.

    \item Aceite feedback construtivo para aprimorar suas habilidades.

    \item Compartilhe experiências relevantes que possam beneficiar os alunos.

    \item Evite comportamentos discriminatórios ou parciais.

    \item Esteja ciente das regras relacionadas à conduta e à ética.

    \item Evite fazer piadas e brincadeiras que possam gerar constrangimento e desconforto nos alunos.
\end{itemize}

\section{Procedimentos e recursos}

\section{Contatos e suporte}

\section{Conlusão}
%O QUE DEVE ESTAR ATENTO NO PRIMEIRO CONTATO COM A TURMA?

%Qual a melhor maneira de interagir com a turma?

%O que fazer em situações de falhas na comunicação ou de conflitos entre alunos e aluno/monitor?

%Procedimento para a verificação da frequência e assimilação dos conteúdos ministrados.